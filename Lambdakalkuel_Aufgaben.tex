\documentclass{scrartcl}
\usepackage[T1]{fontenc}
\usepackage[utf8]{inputenc}
\usepackage[ngerman]{babel}
\usepackage{amsmath,amssymb}
\usepackage{enumerate}
\usepackage{graphicx}

\newcommand{\same}{$\stackrel{\eta,\beta}{=}$}
\renewcommand{\l}{$\lambda$}

\begin{document}
\section*{Aufgaben zum Lambdakalkül}
\subsection*{Aufgabe 1}
Das SKI-Kalkül verwendet die Kombinatoren S = \l x. \l y. \l z. x z (y z), K = \l x. \l y. x und I = \l x. x. Zusätzlich lassen sich die Kombinatoren C = \l f. \l x. \l y. f y x und\\ B = \l f. \l g. \l x. f (g x) definieren.
\begin{enumerate}[1.]
\item Zeigen Sie S K \same{} K I, indem Sie zeigen:\\
S K x y $\Rightarrow^*$ Z und K I x y $\Rightarrow^*$ Z für den gleichen Term Z.
\item Der Kombinator $\Psi$ ist wie folgt definiert: $\Psi$ = \l f. \l g. \l x. \l y. f (g x) (g y).
\begin{enumerate}[(a)]
\item Geben Sie den allgemeinsten Typ von $\Psi$ an.
\item Zeigen Sie C I x y $\Rightarrow^*$ y x.
\item Der Kombinator C' sei definiert durch C' = B (B C) C.\\
Zeigen Sie: C' f x y z $\Rightarrow^*$ f z x y.
\item Der Kombinator S kann dargestellt werden als C' (B ($\Psi$ I) (C I)).\\ Zeigen Sie, dass tatsächlich C' (B ($\Psi$ I) (C I)) x y z $\Rightarrow^*$ x z (y z).
\end{enumerate} 
\end{enumerate}

\end{document}